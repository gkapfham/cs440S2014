%!TEX root=cs440S2014-lab1.tex
% mainfile: cs440S2014-lab1.tex 
\input{labspre.tex}

\usepackage[compact]{titlesec}

\begin{document}
\MYTITLE{Laboratory Assignment One: Customizing and Using the Z Shell}
\MYHEADERS{Laboratory Assignment One}{Due: February 3, 2014}

\section*{Introduction}

% Practicing software engineers normally use a version control system to manage most of the artifacts produced during the
% phases of the software development life cycle.  In this course, we will always use the Git distributed version control
% system to manage the files associated with our laboratory assignments.  In this laboratory assignment, you will learn
% how to use the Bitbucket service for managing Git repositories and the {\tt git} command-line tool in the Ubuntu Linux
% operating system.

Computer scientists who use, configure, and implement an operating system (OS) often interaction with the OS through the use of
the shell.  There are many operating system shells available for the Unix, Linux, and Mac OSX operating systems: sh, tcsh, bash,
and fish.  In this laboratory session, you will learn how to configure and use zsh, arguably the most advanced and customizable
shell for the Linux operating system that is described at \url{http://zsh.sourceforge.net}.

\section*{Configuring the Z Shell}

To start using the Z shell, hereafter abbreviated as zsh, you must open a terminal window and type {\tt zsh}.  Try to navigate
your file system and run a program through zsh.  Do you notice any differences between this new shell and the one that you were
using previously?  What is the name of the shell that is the default for the Ubuntu operating system? Which do you like better?
Why?

\section*{Creating a New Repository}

Now that you have learned how to clone an existing Git repository, you and your team members are responsible for
creating a new repository that contains the source code of a presentation explaining the use of Git. Using screen shots
and easy-to-understand points (and the presentation system in the cloned repository), your presentation should explain
how to use the following Git commands. 

\begin{enumerate} 
			
	\item {\tt git init}

	\item {\tt git status}

	\item {\tt git add} 

	\item {\tt git commit}

	\item {\tt git push}

	\item {\tt git pull} 

	\item One additional {\tt git} command

\end{enumerate}

When you presentation describes a specific Git command, it should explain its input and output with concrete examples.
Your team is responsible for creating one presentation in a manner that ensures each member can make a substantial
contribution. You should use a version control repository to coordinate your work on the presentation.  Students who
would like to learn more about Git can consult Web sites like \url{http://try.github.io/} and
\url{http://gitimmersion.com/}. 

To create a new Git repository that is hosted on the Bitbucket servers, a member of your team should first create a
local directory and then initialize it as a local Git repository.  Next, you should use the Bitbucket
Web site to create a repository that has the same name as the local directory and local repository.  Next, you must
follow Bitbucket's instructions to push the code and tags in your local repository to the one hosted by Bitbucket's
servers.  After completing this step, the chosen team member should share the repository with both the course instructor
and everyone else on the team.  At this point, all members of the team will be able to clone the repository and
manipulate the files stored inside of it.  Now, you must work together to finish the required presentation!

\section*{Summary of the Required Deliverables}

This assignment invites your team to submit one printed version of a tutorial that contains:

\begin{enumerate}
	
	\item A description of the steps that a user must take to configure Git and Bitbucket

	\item A description of the inputs, outputs, and behavior of the aforementioned Git commands

\end{enumerate}

You must also ensure that the instructor has read access to your Bitbucket repository that is named according to the
convention {\tt cs290F2013-lab1-team{\em k}}, with {\tt {\em k}} representing the number of your assigned team. Please
see the instructor if you would like to print your tutorial slides in color.

\end{document}
