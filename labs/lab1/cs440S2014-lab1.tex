%!TEX root=cs440S2014-lab1.tex
% mainfile: cs440S2014-lab1.tex 
\input{labspre.tex}

\usepackage[compact]{titlesec}

\begin{document}
\MYTITLE{Laboratory Assignment One: Customizing and Using the Z Shell}
\MYHEADERS{Laboratory Assignment One}{Due: February 3, 2014}

\section*{Introduction}

% Practicing software engineers normally use a version control system to manage most of the artifacts produced during the
% phases of the software development life cycle.  In this course, we will always use the Git distributed version control
% system to manage the files associated with our laboratory assignments.  In this laboratory assignment, you will learn
% how to use the Bitbucket service for managing Git repositories and the {\tt git} command-line tool in the Ubuntu Linux
% operating system.

Computer scientists who use, configure, and implement an operating system (OS) often interaction with the OS through the use of
the shell.  There are many operating system shells available for the Unix, Linux, and Mac OSX operating systems: sh, tcsh, bash,
and fish.  In this laboratory session, you will learn how to configure and use zsh, arguably the most advanced and customizable
shell for the Linux operating system that is described at \url{http://zsh.sourceforge.net}.

\section*{Configuring and Using the Z Shell}

To start using the Z shell, hereafter abbreviated as zsh, you must open a terminal window and type {\tt zsh}.  Try to navigate
your file system and run a program through zsh.  Do you notice any differences between this new shell and the one that you were
using previously?  What is the name of the shell that is the default for the Ubuntu operating system? Which do you like better?
Why?

Zsh can be quickly and easily configured with a community-driven configuration framework called oh-my-zsh.  You can learn more
about this framework by visiting the following Web site: \url{https://github.com/robbyrussell/oh-my-zsh}. Once you understand the
basics of oh-my-zsh, please follow instructions on the project's Web site to install it in your home account. After installing
this framework, does your shell operate in a different way? If yes, then how?

The oh-my-zsh framework supports a wide variety of themes for your shell.  Try out the different themes and pick the one that you
think is best.  Why did you pick the one that you did? What does it look like? Oh-my-zsh also furnishes a wide variety of plugins.
As part of this laboratory assignment, you should configure zsh to run the git, web-search, z, and zsh-syntax-highlighting
plugins.  After modifying the .zshrc file to use each of the aforementioned plugins, you should try them out and write a brief
commentary on the features that they provide.

Now, you should further customize your Z shell by changing the command prompt so that it includes, at minimum, your username and
the hostname of the computer you are currently using.  Please note that while some themes may include this information, others may
not and thus require additional enhancements.  Students can earn extra credit on this assignment if they configure zsh to display
information about the Git repository that is contained within their current directory; if you are interested in pursuing this
extra credit, please ask the to see instructor's zsh prompt.


\section*{Summary of the Required Deliverables}

This assignment invites your team to submit one printed version of a tutorial that contains:

\begin{enumerate}
	
	\item A description of the steps that a user must take to configure Git and Bitbucket

	\item A description of the inputs, outputs, and behavior of the aforementioned Git commands

\end{enumerate}

You must also ensure that the instructor has read access to your Bitbucket repository that is named according to the
convention {\tt cs290F2013-lab1-team{\em k}}, with {\tt {\em k}} representing the number of your assigned team. Please
see the instructor if you would like to print your tutorial slides in color.

\end{document}
