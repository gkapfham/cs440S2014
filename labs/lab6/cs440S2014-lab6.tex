%!TEX root=cs440S2014-lab6.tex 
% mainfile: cs440S2014-lab6.tex 

%!TEX root=cs440S2014-lab7.tex
% mainfile: cs440S2014-lab7.tex 
% CS 580 style
% Typical usage (all UPPERCASE items are optional):
%       \input 580pre
%       \begin{document}
%       \MYTITLE{Title of document, e.g., Lab 1\\Due ...}
%       \MYHEADERS{short title}{other running head, e.g., due date}
%       \PURPOSE{Description of purpose}
%       \SUMMARY{Very short overview of assignment}
%       \DETAILS{Detailed description}
%         \SUBHEAD{if needed} ...
%         \SUBHEAD{if needed} ...
%          ...
%       \HANDIN{What to hand in and how}
%       \begin{checklist}
%       \item ...
%       \end{checklist}
% There is no need to include a "\documentstyle."
% However, there should be an "\end{document}."
%
%===========================================================
\documentclass[11pt,twoside,titlepage]{article}
%%NEED TO ADD epsf!!
\usepackage{threeparttop}
\usepackage{graphicx}
\usepackage{latexsym}
\usepackage{color}
\usepackage{listings}
\usepackage{fancyvrb}
%\usepackage{pgf,pgfarrows,pgfnodes,pgfautomata,pgfheaps,pgfshade}
\usepackage{tikz}
\usepackage[normalem]{ulem}
\tikzset{
    %Define standard arrow tip
%    >=stealth',
    %Define style for boxes
    oval/.style={
           rectangle,
           rounded corners,
           draw=black, very thick,
           text width=6.5em,
           minimum height=2em,
           text centered},
    % Define arrow style
    arr/.style={
           ->,
           thick,
           shorten <=2pt,
           shorten >=2pt,}
}
\usepackage[noend]{algorithmic}
\usepackage[noend]{algorithm}
\newcommand{\bfor}{{\bf for\ }}
\newcommand{\bthen}{{\bf then\ }}
\newcommand{\bwhile}{{\bf while\ }}
\newcommand{\btrue}{{\bf true\ }}
\newcommand{\bfalse}{{\bf false\ }}
\newcommand{\bto}{{\bf to\ }}
\newcommand{\bdo}{{\bf do\ }}
\newcommand{\bif}{{\bf if\ }}
\newcommand{\belse}{{\bf else\ }}
\newcommand{\band}{{\bf and\ }}
\newcommand{\breturn}{{\bf return\ }}
\newcommand{\mod}{{\rm mod}}
\renewcommand{\algorithmiccomment}[1]{$\rhd$ #1}
\newenvironment{checklist}{\par\noindent\hspace{-.25in}{\bf Checklist:}\renewcommand{\labelitemi}{$\Box$}%
\begin{itemize}}{\end{itemize}}
\pagestyle{threepartheadings}
\usepackage{url}
\usepackage{wrapfig}
% removing the standard hyperref to avoid the horrible boxes
%\usepackage{hyperref}
\usepackage[hidelinks]{hyperref}
% added in the dtklogos for the bibtex formatting
\usepackage{dtklogos}
%=========================
% One-inch margins everywhere
%=========================
\setlength{\topmargin}{0in}
\setlength{\textheight}{8.5in}
\setlength{\oddsidemargin}{0in}
\setlength{\evensidemargin}{0in}
\setlength{\textwidth}{6.5in}
%===============================
%===============================
% Macro for document title:
%===============================
\newcommand{\MYTITLE}[1]%
   {\begin{center}
     \begin{center}
     \bf
     CMPSC 440\\Operating Systems\\
     Spring 2014
     \medskip
     \end{center}
     \bf
     #1
     \end{center}
}
%================================
% Macro for headings:
%================================
\newcommand{\MYHEADERS}[2]%
   {\lhead{#1}
    \rhead{#2}
    %\immediate\write16{}
    %\immediate\write16{DATE OF HANDOUT?}
    %\read16 to \dateofhandout
    \def \dateofhandout {March 31, 2014}
    \lfoot{\sc Handed out on \dateofhandout}
    %\immediate\write16{}
    %\immediate\write16{HANDOUT NUMBER?}
    %\read16 to\handoutnum
    \def \handoutnum {8}
    \rfoot{Handout \handoutnum}
   }

%================================
% Macro for bold italic:
%================================
\newcommand{\bit}[1]{{\textit{\textbf{#1}}}}

%=========================
% Non-zero paragraph skips.
%=========================
\setlength{\parskip}{1ex}

%=========================
% Create various environments:
%=========================
\newcommand{\PURPOSE}{\par\noindent\hspace{-.25in}{\bf Purpose:\ }}
\newcommand{\SUMMARY}{\par\noindent\hspace{-.25in}{\bf Summary:\ }}
\newcommand{\DETAILS}{\par\noindent\hspace{-.25in}{\bf Details:\ }}
\newcommand{\HANDIN}{\par\noindent\hspace{-.25in}{\bf Hand in:\ }}
\newcommand{\SUBHEAD}[1]{\bigskip\par\noindent\hspace{-.1in}{\sc #1}\\}
%\newenvironment{CHECKLIST}{\begin{itemize}}{\end{itemize}}


\usepackage[compact]{titlesec}

\begin{document} \MYTITLE{Laboratory Assignment Six: Measuring the Size of Program Variables in Java and C}
\MYHEADERS{Laboratory Assignment Six}{Due: March 31, 2014}

% \vspace*{-.32in}

\section*{Introduction}

  Main memory, in the form of random-access memory (RAM), is an important component of a computer.  The memory management
  unit (MMU) of an operating system is responsible for allocating, deallocating, and compressing the memory required by
  the programs that the operating system is managing. One of the first steps towards understanding the management of
  memory is to learn more about the size of the primitive variables that programs ask the MMU to allocate to memory. In
  this laboratory assignment, you will write a C and a Java program to determine the size of the most prevalently used
  primitive types supported by these two languages. 

\section*{Sizing Variables in a C Program}

  The C programming language provides a {\tt sizeof} method that a program can use to determine the size of a primitive
  variable like an {\tt int}. What are the inputs and outputs of the {\tt sizeof} function? How does the {\tt sizeof}
  function work? You should write a {\tt usesizeof.c} program and use the {\tt gcc} compiler to build a binary called
  {\tt usesizeof}. Your program should calculate the size, in both bits and bytes, of the following variable types: {\tt
  short}, {\tt int}, {\tt long}, {\tt long long}, {\tt float}, {\tt double}, {\tt long double}, {\tt char}, {\tt char*},
  and {\tt \_Bool}. Do the sizes reported by your program make sense? Why?

\section*{Sizing Variables in a Java Program}

  While the Java language specification states that all primitive types should be uniform across implementations of the
  compiler and virtual machine, it is still worthwhile to determine the size of primitives in this language. However,
  the Java language does not provide a {\tt sizeof} method like the one that you used in your C program. To learn more
  about how to calculate the size of a Java primitive, read the article called ``Sizeof for java: object sizing
  revisited'' by Vladimir Roubtsov. In the downloads section of this online article, you will find a Zip file called
  {\tt 02-qa-1226-sizeof.zip}. 

  After downloading this file, decompress it in the directory that you are using to store the files for this laboratory
  assignment. Using the example source code provide with this archive, you should write a Java program called {\tt
  UseSizeOf.java} that calculates the size, in both bits and bytes, of the following variable types:  {\tt short}, {\tt
  int}, {\tt long}, {\tt float}, {\tt double}, {\tt boolean}, and {\tt char}. Don't forget that {\tt UseSizeof.java}
  will not compile and the program will not run unless you have {\tt objectprofiler.jar} in your {\tt CLASSPATH}
  environment variable. 
  
  When you analyze the output of the {\tt UseSizeOf} program, you should take into account that Java's autoboxing
  feature will transform your primitive types into objects.  As such, your final output needs to consider the size of a
  reference pointer. What are the final sizes reported by your program? Do they make sense? Why? How do they compare to
  the sizes of the C variables?

\section*{Analysis of the Sizing Results}

  This laboratory assignment focuses on computing the size of primitive variables. How would the results vary if we
  considered the sizing of pointers, dynamically allocated variables, and heap-resident objects? What do the results
  from running your programs suggest about the trade-offs in the management of memory pages in the entire memory address
  space? To complete this assignment, you should write a short report that explains the programs that you write and
  gives the full output from each program.  Next, you should compare and contrast the results from the C and Java
  programs. Finally, you should make sure that you respond to all of the questions posed in the assignment.

\section*{Summary of the Required Deliverables}

This assignment invites you to submit printed and signed versions of the following deliverables: 

\begin{enumerate}

  \item The source code of the {\tt usesizeof.c} and {\tt UseSizeOf.java} programs

  \item The command lines that you used to compile and run the C and Java programs

  \item A comprehensive report that explains the results and responds to all stated questions 

  \item Suggestions for new features to add to the variable sizing programs

  \item Suggestions for future experiments and analyses that use the variable sizing programs

  \item A reflective discussion of the challenges that you encountered when completing this assignment

\end{enumerate}

In adherence to the honor code, students should complete this assignment on an individual basis. While it is appropriate
for students in this class to have high-level conversations about the assignment, it is necessary to distinguish
carefully between the student who discusses the principles underlying a problem with others and the student who produces
assignments that are identical to, or merely variations on, someone else's work.  As such, deliverables that are nearly
identical to the work of others will be taken as evidence of violating the \mbox{Honor Code}.  

  \end{document}
