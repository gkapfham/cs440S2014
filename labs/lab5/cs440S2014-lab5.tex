%!TEX root=cs440S2014-lab5.tex 
% mainfile: cs440S2014-lab5.tex 

\input{labspre.tex}

\usepackage[compact]{titlesec}

\begin{document} \MYTITLE{Laboratory Assignment Five: Using Simulation to Evaluate Scheduling Algorithms} 
\MYHEADERS{Laboratory Assignment Five}{Due: March 3, 2014}

\section*{Introduction}

  Modern operating systems normally have a kernel that runs a scheduler.  The goal of the scheduler is to pick which
  process will next run on the central processing unit (CPU).  There are a wide variety of scheduling algorithms, such
  as first-come first-served (FCSF), shortest job first (SJF), round-robin (RR), and priority scheduling.  Since these
  algorithms are normally very complex and often difficult to study in an analytical fashion, operating systems
  designers may use simulators to better understand their trade-offs in efficiency and fairness. In this laboratory
  assignment, you will use a scheduling simulator, implemented by Jim Weller, to evaluate several scheduling algorithms.

\section*{Using Simulation to Evaluate Schedulers}

  To learn more about the simulator that we will use to study the scheduling algorithms, please visit the following Web
  site: \url{http://jimweller.com/jim-weller/jim/java_proc_sched/}. Next, use your terminal window to navigate into the
  version control repository that is used for sharing source code and type the command {\tt git pull}.  Please change
  into the {\tt lab5/jimweller/} directory and study the files that are now available. How many files are there? What is
  their purpose?

  The file called {\tt jobList.dat} contains three lines that share information about the execution characteristics of a
  process to be scheduled by the operating system.  For instance, the first line in the file contains three numbers: ``10
  5 1'' with the first value corresponding to the amount of CPU time required by the process and the second and third
  numbers denoting the time that separates the arrival time and priority of the process, respectively. If you carefully
  study {\tt jobList.dat} you will notice that the first simulation will only run a total of three processes.

  To run the simulator, you should type {\tt java -jar cpu.jar} in the terminal window.  What do you now see on your
  screen? To run a simulation, you must first load the {\tt jobList.dat} file.  Then, you should pick a scheduling
  algorithm, indicate the speed at which the algorithm should be animated, and check whether or not the algorithm should
  consider process priority. Now, you can click the check box to run the algorithm. (Please note that the ``check box''
  is really a toggle button that does not render correctly on Ubuntu. If you click this button and the simulator does
  not run as anticipated, then click it again and see if this starts the animation.)

\section*{Summary of the Required Deliverables}

This assignment invites you to submit printed and signed versions of the following deliverables: 

\begin{enumerate}

  \item A clear description of the term semaphore, with comments about the its historical origins 

  \item A revised and extended implementation of the producer-consumer model

  \item A comprehensive analysis of the output of each defective multi-threaded Java program

  \item A reflection of the characteristics of the two different producer-consumer models

\end{enumerate}

In adherence to the honor code, students should complete this assignment on an individual basis. While it is appropriate
for students in this class to have high-level conversations about the assignment, it is necessary to distinguish
carefully between the student who discusses the principles underlying a problem with others and the student who produces
assignments that are identical to, or merely variations on, someone else's work.  As such, deliverables that are nearly
identical to the work of others will be taken as evidence of violating the \mbox{Honor Code}.  



  \end{document}
