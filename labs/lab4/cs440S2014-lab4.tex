%!TEX root=cs440S2014-lab4.tex 
% mainfile: cs440S2014-lab4.tex 

%!TEX root=cs440S2014-lab7.tex
% mainfile: cs440S2014-lab7.tex 
% CS 580 style
% Typical usage (all UPPERCASE items are optional):
%       \input 580pre
%       \begin{document}
%       \MYTITLE{Title of document, e.g., Lab 1\\Due ...}
%       \MYHEADERS{short title}{other running head, e.g., due date}
%       \PURPOSE{Description of purpose}
%       \SUMMARY{Very short overview of assignment}
%       \DETAILS{Detailed description}
%         \SUBHEAD{if needed} ...
%         \SUBHEAD{if needed} ...
%          ...
%       \HANDIN{What to hand in and how}
%       \begin{checklist}
%       \item ...
%       \end{checklist}
% There is no need to include a "\documentstyle."
% However, there should be an "\end{document}."
%
%===========================================================
\documentclass[11pt,twoside,titlepage]{article}
%%NEED TO ADD epsf!!
\usepackage{threeparttop}
\usepackage{graphicx}
\usepackage{latexsym}
\usepackage{color}
\usepackage{listings}
\usepackage{fancyvrb}
%\usepackage{pgf,pgfarrows,pgfnodes,pgfautomata,pgfheaps,pgfshade}
\usepackage{tikz}
\usepackage[normalem]{ulem}
\tikzset{
    %Define standard arrow tip
%    >=stealth',
    %Define style for boxes
    oval/.style={
           rectangle,
           rounded corners,
           draw=black, very thick,
           text width=6.5em,
           minimum height=2em,
           text centered},
    % Define arrow style
    arr/.style={
           ->,
           thick,
           shorten <=2pt,
           shorten >=2pt,}
}
\usepackage[noend]{algorithmic}
\usepackage[noend]{algorithm}
\newcommand{\bfor}{{\bf for\ }}
\newcommand{\bthen}{{\bf then\ }}
\newcommand{\bwhile}{{\bf while\ }}
\newcommand{\btrue}{{\bf true\ }}
\newcommand{\bfalse}{{\bf false\ }}
\newcommand{\bto}{{\bf to\ }}
\newcommand{\bdo}{{\bf do\ }}
\newcommand{\bif}{{\bf if\ }}
\newcommand{\belse}{{\bf else\ }}
\newcommand{\band}{{\bf and\ }}
\newcommand{\breturn}{{\bf return\ }}
\newcommand{\mod}{{\rm mod}}
\renewcommand{\algorithmiccomment}[1]{$\rhd$ #1}
\newenvironment{checklist}{\par\noindent\hspace{-.25in}{\bf Checklist:}\renewcommand{\labelitemi}{$\Box$}%
\begin{itemize}}{\end{itemize}}
\pagestyle{threepartheadings}
\usepackage{url}
\usepackage{wrapfig}
% removing the standard hyperref to avoid the horrible boxes
%\usepackage{hyperref}
\usepackage[hidelinks]{hyperref}
% added in the dtklogos for the bibtex formatting
\usepackage{dtklogos}
%=========================
% One-inch margins everywhere
%=========================
\setlength{\topmargin}{0in}
\setlength{\textheight}{8.5in}
\setlength{\oddsidemargin}{0in}
\setlength{\evensidemargin}{0in}
\setlength{\textwidth}{6.5in}
%===============================
%===============================
% Macro for document title:
%===============================
\newcommand{\MYTITLE}[1]%
   {\begin{center}
     \begin{center}
     \bf
     CMPSC 440\\Operating Systems\\
     Spring 2014
     \medskip
     \end{center}
     \bf
     #1
     \end{center}
}
%================================
% Macro for headings:
%================================
\newcommand{\MYHEADERS}[2]%
   {\lhead{#1}
    \rhead{#2}
    %\immediate\write16{}
    %\immediate\write16{DATE OF HANDOUT?}
    %\read16 to \dateofhandout
    \def \dateofhandout {March 31, 2014}
    \lfoot{\sc Handed out on \dateofhandout}
    %\immediate\write16{}
    %\immediate\write16{HANDOUT NUMBER?}
    %\read16 to\handoutnum
    \def \handoutnum {8}
    \rfoot{Handout \handoutnum}
   }

%================================
% Macro for bold italic:
%================================
\newcommand{\bit}[1]{{\textit{\textbf{#1}}}}

%=========================
% Non-zero paragraph skips.
%=========================
\setlength{\parskip}{1ex}

%=========================
% Create various environments:
%=========================
\newcommand{\PURPOSE}{\par\noindent\hspace{-.25in}{\bf Purpose:\ }}
\newcommand{\SUMMARY}{\par\noindent\hspace{-.25in}{\bf Summary:\ }}
\newcommand{\DETAILS}{\par\noindent\hspace{-.25in}{\bf Details:\ }}
\newcommand{\HANDIN}{\par\noindent\hspace{-.25in}{\bf Hand in:\ }}
\newcommand{\SUBHEAD}[1]{\bigskip\par\noindent\hspace{-.1in}{\sc #1}\\}
%\newenvironment{CHECKLIST}{\begin{itemize}}{\end{itemize}}


\usepackage[compact]{titlesec}

\begin{document} \MYTITLE{Laboratory Assignment Four: A Producer-Consumer Model with Semaphores}
\MYHEADERS{Laboratory Assignment Four}{Due: February 24, 2014}

\section*{Introduction}

Processes and threads are commonly used by the operating system itself and by the programs that run on the operating
system. In the previous laboratory assignment you learned how to implement and evaluate Java programs that use the {\tt
  synchronized} keyword to create a multi-threaded produced consumer model. In this laboratory assignment, you will
download, use, extend, and experiment with a multi-threaded producer-consumer model that uses semaphores,  As in the
previous assignment, you will add some features to the model that enable you to experiment with it more effectively.
Finally, you will systematically break various parts of the model in order to observe what can go wrong when
implementing multi-threaded Java programs that use semaphores.

\section*{Accessing the Producer-Consumer Model}

As in the previous assignment, you should change into directory for the Git repository that I use to share files with
you. Now, you can type the command {\tt git pull} to retrieve all of the files for this laboratory assignment. If this
step did not work correctly, then please see the instructor.  Finally, you should use your terminal window to browse
the files that are in this Git repository.  In particular, please look in the {\tt labs/lab4/src/} directory and use Vim
to study the Java programs that you find. What files are used to implement the producer-consumer model?

\section*{Understanding and Extending the Producer-Consumer Model}

In this laboratory session, we will use code segments from Vijay K.\ Garg's book, {\em Concurrent and Distributed
  Computing in Java}, to learn more about how to implement a producer-consumer model using semaphores.  Before you start
to modify the programs that you received through the Git repository, take some time to learn more about the history of
the term semaphore.  What is the meaning of this term when it is used in a context not connected to the discipline of
computer science? What was the first paper to propose the idea of the semaphore? Finally, it is important to note that
this program uses both binary and counting semaphores. What are the similarities and differences between these two types
of semaphores?

After you have carefully studied the source code of the producer-consumer model, you can compile and execute it.  What
type of output does this program produce?  Will this program halt?  If yes, then how long will it take to halt?  If no,
then why does it not halt? How is this program similar to and different from the producer-consumer model that you used
in the last assignment?

\begin{sloppypar} 
  As you will see from studying the source code, the current implementation of the {\tt
    ProducerConsumer} always produces debugging output. Leveraging the JCommander system that you learned how to use in
  a previous laboratory assignment, you should now implement a command-line interface for the {\tt ProducerConsumer}.
  First, you should add a parameter that can indicate whether or not the {\tt ProducerConsumer} should produce debugging
  output.  To implement debugging output correctly, you will need to find each of the classes that execute
  output-producing statements.  Next, you should notice that the current implementation of the {\tt ProducerConsumer}
  only creates a single producer and consumer.  To extend the program, you should add a command-line argument that
  allows the user to specify the number of consumers.  
\end{sloppypar}

Finally, you should notice that the {\tt ProducerConsumer} does not does not furnish any information about the specific
thread that is producing the debugging output.  As such, you should further extend the debugging output so that the name
of the thread is also displayed. To add this feature, you may need to change the inheritance hierarchy for the {\tt
  Producer} and {\tt Consumer} classes.  You will also need to determine which methods provided by {\tt
  java.lang.Thread} return information about the name of the thread that is currently executing.

\section*{Understanding Defects in Multi-Threaded Programs}

After you have finished writing the source code to support the command-line arguments for the {\tt ProducerConsumer},
you can take steps to explore what happens if you make mistakes when implementing multi-threaded Java programs that use
semaphores.  Please ensure that you carefully make the following changes, recompile the program, execute the program,
record the incorrect program output, and comment on why the output is evident.  Once you are done making the required
defective implementations of the program, please return the modified code to the correct state!

\begin{enumerate} 
 
  \item Comment out the first two calls to {\tt P} in both the {\tt deposit} and {\tt fetch} methods.  For example, in
    the {\tt deposit} method you would produce the following new source code statements.  What output does the program
    produce?  Why does it produce this output?  

    \begin{verbatim}
       //isFull.P(); // wait if buffer is full
       //mutex.P(); // ensures mutual exclusion
    \end{verbatim} 

\end{enumerate}


\section*{Summary of the Required Deliverables}

This assignment invites you to submit printed and signed versions of the following deliverables: 

\begin{enumerate} \item A paragraph that explains how multi-threaded Java programs use {\tt synchronized} \item A
    detailed discussion of the execution behavior of a multi-threaded Java program \item A comprehensive analysis of the
    output of each defective multi-threaded Java program \item The report from an experimental study that characterizes
    the performance of the model \end{enumerate}

Students are strongly encouraged to write their laboratory report in \LaTeX~and use tools such as the R language for
statistical computation to better analyze and visualize the empirical results. Please see the instructor if you have
questions about the requirements for these deliverables.

In adherence to the honor code, students should complete this assignment on an individual basis. While it is appropriate
for students in this class to have high-level conversations about the assignment, it is necessary to distinguish
carefully between the student who discusses the principles underlying a problem with others and the student who produces
assignments that are identical to, or merely variations on, someone else's work.  As such, deliverables that are nearly
identical to the work of others will be taken as evidence of violating the \mbox{Honor Code}.  



  \end{document}
