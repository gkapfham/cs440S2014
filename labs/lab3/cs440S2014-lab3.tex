%!TEX root=cs440S2014-lab3.tex
% mainfile: cs440S2014-lab3.tex 
\input{labspre.tex}

\usepackage[compact]{titlesec}

\begin{document}
\MYTITLE{Laboratory Assignment Three: Using a Multi-Threaded Producer-Consumer Model}
\MYHEADERS{Laboratory Assignment Three}{Due: February 17, 2014}

\section*{Introduction}

Processes and threads are commonly used by the operating system itself and the programs that run on the operating system.  In this
laboratory assignment, you will download, use, extend, and experiment with a multi-threaded producer-consumer model.  The
producer-consumer model, sometimes known as the bounded-buffer model, illustrates non-deterministic execution of threads that
share a common, fixed-size buffer.  In this assignment, you will download and use a working version of this model.  Then, you will
add some features to the model to enable you to experiment with it more effectively.  Finally, you will systematically break
various parts of the model in order to observe what can go wrong when implementing multi-threaded programs.

\section*{File System Traversal}

\section*{Experimental Study}

\section*{Summary of the Required Deliverables}

This assignment invites you to submit printed and signed versions of the following deliverables: 

  \begin{enumerate}
    \item The source code of the file system traversal tool that you implemented in the Java language
    \item The output of the file system traversal tool when run on several small file system regions
    \item The report from an experimental study that characterizes the use of the Linux file system
  \end{enumerate}

Students are strongly encouraged to write their laboratory report in \LaTeX~and use tools such as the R language for statistical
computation and Graphviz to better view and understand the empirical results. Please see the instructor if you have questions
about these deliverables.

In adherence to the honor code, students should complete this assignment on an individual basis. While it is appropriate for
students in this class to have high-level conversations about the assignment, it is necessary to distinguish carefully between the
student who discusses the principles underlying a problem with others and the student who produces assignments that are identical
to, or merely variations on, someone else's work.  As such, deliverables that are nearly identical to the work of others will be
taken as evidence of violating the \mbox{Honor Code}.  



  \end{document}
