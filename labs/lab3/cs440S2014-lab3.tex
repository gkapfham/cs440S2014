%!TEX root=cs440S2014-lab3.tex
% mainfile: cs440S2014-lab3.tex 
%!TEX root=cs440S2014-lab7.tex
% mainfile: cs440S2014-lab7.tex 
% CS 580 style
% Typical usage (all UPPERCASE items are optional):
%       \input 580pre
%       \begin{document}
%       \MYTITLE{Title of document, e.g., Lab 1\\Due ...}
%       \MYHEADERS{short title}{other running head, e.g., due date}
%       \PURPOSE{Description of purpose}
%       \SUMMARY{Very short overview of assignment}
%       \DETAILS{Detailed description}
%         \SUBHEAD{if needed} ...
%         \SUBHEAD{if needed} ...
%          ...
%       \HANDIN{What to hand in and how}
%       \begin{checklist}
%       \item ...
%       \end{checklist}
% There is no need to include a "\documentstyle."
% However, there should be an "\end{document}."
%
%===========================================================
\documentclass[11pt,twoside,titlepage]{article}
%%NEED TO ADD epsf!!
\usepackage{threeparttop}
\usepackage{graphicx}
\usepackage{latexsym}
\usepackage{color}
\usepackage{listings}
\usepackage{fancyvrb}
%\usepackage{pgf,pgfarrows,pgfnodes,pgfautomata,pgfheaps,pgfshade}
\usepackage{tikz}
\usepackage[normalem]{ulem}
\tikzset{
    %Define standard arrow tip
%    >=stealth',
    %Define style for boxes
    oval/.style={
           rectangle,
           rounded corners,
           draw=black, very thick,
           text width=6.5em,
           minimum height=2em,
           text centered},
    % Define arrow style
    arr/.style={
           ->,
           thick,
           shorten <=2pt,
           shorten >=2pt,}
}
\usepackage[noend]{algorithmic}
\usepackage[noend]{algorithm}
\newcommand{\bfor}{{\bf for\ }}
\newcommand{\bthen}{{\bf then\ }}
\newcommand{\bwhile}{{\bf while\ }}
\newcommand{\btrue}{{\bf true\ }}
\newcommand{\bfalse}{{\bf false\ }}
\newcommand{\bto}{{\bf to\ }}
\newcommand{\bdo}{{\bf do\ }}
\newcommand{\bif}{{\bf if\ }}
\newcommand{\belse}{{\bf else\ }}
\newcommand{\band}{{\bf and\ }}
\newcommand{\breturn}{{\bf return\ }}
\newcommand{\mod}{{\rm mod}}
\renewcommand{\algorithmiccomment}[1]{$\rhd$ #1}
\newenvironment{checklist}{\par\noindent\hspace{-.25in}{\bf Checklist:}\renewcommand{\labelitemi}{$\Box$}%
\begin{itemize}}{\end{itemize}}
\pagestyle{threepartheadings}
\usepackage{url}
\usepackage{wrapfig}
% removing the standard hyperref to avoid the horrible boxes
%\usepackage{hyperref}
\usepackage[hidelinks]{hyperref}
% added in the dtklogos for the bibtex formatting
\usepackage{dtklogos}
%=========================
% One-inch margins everywhere
%=========================
\setlength{\topmargin}{0in}
\setlength{\textheight}{8.5in}
\setlength{\oddsidemargin}{0in}
\setlength{\evensidemargin}{0in}
\setlength{\textwidth}{6.5in}
%===============================
%===============================
% Macro for document title:
%===============================
\newcommand{\MYTITLE}[1]%
   {\begin{center}
     \begin{center}
     \bf
     CMPSC 440\\Operating Systems\\
     Spring 2014
     \medskip
     \end{center}
     \bf
     #1
     \end{center}
}
%================================
% Macro for headings:
%================================
\newcommand{\MYHEADERS}[2]%
   {\lhead{#1}
    \rhead{#2}
    %\immediate\write16{}
    %\immediate\write16{DATE OF HANDOUT?}
    %\read16 to \dateofhandout
    \def \dateofhandout {March 31, 2014}
    \lfoot{\sc Handed out on \dateofhandout}
    %\immediate\write16{}
    %\immediate\write16{HANDOUT NUMBER?}
    %\read16 to\handoutnum
    \def \handoutnum {8}
    \rfoot{Handout \handoutnum}
   }

%================================
% Macro for bold italic:
%================================
\newcommand{\bit}[1]{{\textit{\textbf{#1}}}}

%=========================
% Non-zero paragraph skips.
%=========================
\setlength{\parskip}{1ex}

%=========================
% Create various environments:
%=========================
\newcommand{\PURPOSE}{\par\noindent\hspace{-.25in}{\bf Purpose:\ }}
\newcommand{\SUMMARY}{\par\noindent\hspace{-.25in}{\bf Summary:\ }}
\newcommand{\DETAILS}{\par\noindent\hspace{-.25in}{\bf Details:\ }}
\newcommand{\HANDIN}{\par\noindent\hspace{-.25in}{\bf Hand in:\ }}
\newcommand{\SUBHEAD}[1]{\bigskip\par\noindent\hspace{-.1in}{\sc #1}\\}
%\newenvironment{CHECKLIST}{\begin{itemize}}{\end{itemize}}


\usepackage[compact]{titlesec}

\begin{document}
\MYTITLE{Laboratory Assignment Three: Using a Multi-Threaded Producer-Consumer Model}
\MYHEADERS{Laboratory Assignment Three}{Due: February 17, 2014}

\section*{Introduction}

Processes and threads are commonly used by the operating system itself and the programs that run on the operating system.  In this
laboratory assignment, you will download, use, extend, and experiment with a multi-threaded producer-consumer model.  The
producer-consumer model, sometimes known as the bounded-buffer model, illustrates non-deterministic execution of threads that
share a common, fixed-size buffer.  In this assignment, you will download and use a working version of this model.  Then, you will
add some features to the model to enable you to experiment with it more effectively.  Finally, you will systematically break
various parts of the model in order to observe what can go wrong when implementing multi-threaded programs.

\section*{Accessing the Producer-Consumer Model}

During this laboratory assignment, and some subsequent assignments, we will securely communicate with the Bitbucket.org servers
that will host the source code for our assignment. Students who are already comfortable with using Git and Bitbucket may skip the
majority of these steps and simply ask the instructor to share the course repository with them.  For those students who are not
yet using Git and Bitbucket, this laboratory assignment will teach you all of the steps needed to configure the accounts on the
departmental servers and the Bitbucket service.  Throughout the assignment, you should refer to the following Web site for
additional information about the use of Git and Bitbucket: \url{https://confluence.atlassian.com/display/BITBUCKET/Bitbucket+101}.  

\begin{enumerate}
	
  \item If you have never done so before, you must use the {\tt ssh-keygen} program to create secure-shell keys that you
    can use to support your communication with the Bitbucket servers.  Type {\tt man ssh-keygen} and talk with the
    members of your team to learn more about how to use this program.  What files does {\tt ssh-keygen} produce?  Where
    does this program store these files?

  \item If you do not already have a Bitbucket account, please go to the Bitbucket Web site and create one --- 
    make sure that you use your {\tt allegheny.edu} email address so that you can create an unlimited number of free
    Bitbucket repositories. Then, upload your ssh key to Bitbucket.

  \item Now, you need to test to see if you can authenticate with the Bitbucket servers.  First, show the course instructor that
    you have correctly configured your Bitbucket account.  Now, ask the instructor to share the course's Git repository with you.
    Open a terminal window on your workstation and change into the directory where you will store your files for this course.  For
    instance, you might have made a {\tt cs440S2014/} directory that will contain the Git repository that I will always use to
    share files with you.  Once you have done so and only after I have shared the Git repository with you, please type the
    following command: {\tt git clone git@bitbucket.org:gkapfham/cs440s2014-share.git}.  If everything worked correctly, you
    should be able to download all of the files that you will need to use for this laboratory assignment. Please resolve any
    problems that you encountered by first reviewing the Bitbucket documentation and then discussing the matter with your team. If
    you are still not able to run the {\tt git clone} command, then please see the instructor.

  \item Using your terminal window, you should browse the files that are in this Git repository.  In particular, please look in
    the {\tt labs/lab3/src/} directory and use Vim to study the two Java programs that you find. What files are available for the
    Producer-Consumer model?

\end{enumerate}

\section*{Understanding and Extending the Producer-Consumer Model}

After you have carefully studied the source code of the Producer-Consumer model, you should compile and execute it.  What type of
output does this program produce?  Will this program halt?  If yes, then how long will it take to halt?  If no, then why does it
not halt? Finally, you should use the {\tt /usr/bin/time} program to time how long it takes to run the Producer-Consumer.

In order to better understanding how this program works, you should run it and then use the {\tt ps -eLf} and {\tt ps aux}
commands to track what Java processes are started when the model executes. You should also use the {\tt gnome-system-monitor} to
learn more about this program's behavior (to run this program you can type {\tt monitor} in the Unity dash).  To the best of your
ability, you should use these three programs --- and any others that you deem to be useful and relevant --- to better understand
how the Producer-Consumer model creates processes and/or threads.

As you will see from studying the source code, the current implementation of the Producer-Consumer model has several hard-coded
variables that control its behavior. Leveraging the JCommander project that you learned how to use in a previous laboratory
assignment, you should now implement a command-line interface for the Producer-Consumer model.  This interface should make it
possible to specify whether or not debugging output must be produced, the number of data items that must be produced and consumed,
and the total number of consumers.  Details about the command-line arguments are provided in the comments of the {\tt
ProducerConsumerModel.java} file. 

\section*{Experimental Study of Performance}

\section*{Summary of the Required Deliverables}

This assignment invites you to submit printed and signed versions of the following deliverables: 

  \begin{enumerate}
    \item The source code of the file system traversal tool that you implemented in the Java language
    \item The output of the file system traversal tool when run on several small file system regions
    \item The report from an experimental study that characterizes the use of the Linux file system
  \end{enumerate}

Students are strongly encouraged to write their laboratory report in \LaTeX~and use tools such as the R language for statistical
computation and Graphviz to better view and understand the empirical results. Please see the instructor if you have questions
about these deliverables.

In adherence to the honor code, students should complete this assignment on an individual basis. While it is appropriate for
students in this class to have high-level conversations about the assignment, it is necessary to distinguish carefully between the
student who discusses the principles underlying a problem with others and the student who produces assignments that are identical
to, or merely variations on, someone else's work.  As such, deliverables that are nearly identical to the work of others will be
taken as evidence of violating the \mbox{Honor Code}.  



  \end{document}
