%!TEX root=cs440S2014-lab7.tex 
% mainfile: cs440S2014-lab7.tex 

\input{labspre.tex}

\usepackage[compact]{titlesec}

\begin{document} \MYTITLE{Laboratory Assignment Seven: Evaluating the Efficiency of File System Searching}
\MYHEADERS{Laboratory Assignment Seven}{Due: April 7, 2014}

% \vspace*{-.32in}

\section*{Introduction}

Since all computer applications ultimately need to store and retrieve information, it is essential that an operating
system  provide a file system to support these activities.  It is also important to ensure that application programs
effectively use the primitives provided by the file system.  In this laboratory assignment, we will explore and
experimentally evaluate two file system searching tools, {\tt grep} and {\tt ag}. The first tool that we will study,
{\tt grep}, was implemented years ago by Ken Thompson, one of the implementors of Unix.  The second tool, known as {\tt
ag} or the ``Silver Searcher'', is a new tool for file system searching. As part of this assignment, you will learn the
command lines associated with both of these tools and then design a benchmarking framework to determine which one is
faster.  Finally, you will explore at least one advanced application of these tools.

\section*{File System Searching with {\tt grep}, {\tt ag}, and {\tt nautilus}}

Before you start using these two tools, you should learn about their history. For instance, you should learn exactly
when the tools were implemented and who initially created them. Next, you should run the {\tt man} command in your
terminal window and study the output to ensure that you understand the basic command lines that both tools support. What
are the basic commands for searching a specific directory with these two tools? Finally, you should type {\tt nautilus}
in your terminal window to run Ubuntu's graphical file system browser.  How does {\tt nautilus} support file
system searching?  Which tool seems to furnish the most robust set of searching features?

\section*{Experimentally Evaluating File System Searchers}

As you are implementing a large software system, like an operating system or a compiler, it is important for you to have
the ability to quickly search through a large collection of files for those that contain a specified word or pattern.
For this part of the laboratory assignment, you should find five separate directories of source code that contain a
substantial number of files. Next, you should use a Linux command to determine how many files are in each of the chosen
directories; please make sure that you clearly explain how you counted the total number of files.

The next step in the experimental evaluation of {\tt grep} and {\tt ag} is to devise at least five different command
lines that you can run for each of the tools. Then, for each of the five commands and the five different directories
containing code, you should evaluate the efficiency and effectiveness of the tools. In particular, you should evaluate
the searchers according to the following metrics: (i) the number of results produced, (ii) the usefulness and accuracy
of the output, and (iii) the speed at which the tools produce their output. When conducting the experiments to evaluate
efficiency, please run multiple trials and report arithmetic means and standard deviations. 

\section*{Analysis of the Sizing Results}

\section*{Advanced Use of File System Searchers}

\section*{Summary of the Required Deliverables}

This assignment invites you to submit printed and signed versions of the following deliverables: 

\begin{enumerate}

  \item 

  \item The command lines that you used run all of your benchmarks with {\tt grep} and {\tt ag} 

  \item A comprehensive report that explains the results and responds to all of the stated questions 

  \item The source code and documentation for an advanced use of the file system searching tools

  \item Ideas for future experiments and analyses that use the file system searchers 

  \item A reflective discussion of the challenges that you encountered when completing this assignment

\end{enumerate}

In adherence to the honor code, students should complete this assignment on an individual basis. While it is appropriate
for students in this class to have high-level conversations about the assignment, it is necessary to distinguish
carefully between the student who discusses the principles underlying a problem with others and the student who produces
assignments that are identical to, or merely variations on, someone else's work.  As such, deliverables that are nearly
identical to the work of others will be taken as evidence of violating the \mbox{Honor Code}.  

  \end{document}
