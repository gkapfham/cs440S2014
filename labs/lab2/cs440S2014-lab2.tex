%!TEX root=cs440S2014-lab2.tex
% mainfile: cs440S2014-lab2.tex 
%!TEX root=cs440S2014-lab7.tex
% mainfile: cs440S2014-lab7.tex 
% CS 580 style
% Typical usage (all UPPERCASE items are optional):
%       \input 580pre
%       \begin{document}
%       \MYTITLE{Title of document, e.g., Lab 1\\Due ...}
%       \MYHEADERS{short title}{other running head, e.g., due date}
%       \PURPOSE{Description of purpose}
%       \SUMMARY{Very short overview of assignment}
%       \DETAILS{Detailed description}
%         \SUBHEAD{if needed} ...
%         \SUBHEAD{if needed} ...
%          ...
%       \HANDIN{What to hand in and how}
%       \begin{checklist}
%       \item ...
%       \end{checklist}
% There is no need to include a "\documentstyle."
% However, there should be an "\end{document}."
%
%===========================================================
\documentclass[11pt,twoside,titlepage]{article}
%%NEED TO ADD epsf!!
\usepackage{threeparttop}
\usepackage{graphicx}
\usepackage{latexsym}
\usepackage{color}
\usepackage{listings}
\usepackage{fancyvrb}
%\usepackage{pgf,pgfarrows,pgfnodes,pgfautomata,pgfheaps,pgfshade}
\usepackage{tikz}
\usepackage[normalem]{ulem}
\tikzset{
    %Define standard arrow tip
%    >=stealth',
    %Define style for boxes
    oval/.style={
           rectangle,
           rounded corners,
           draw=black, very thick,
           text width=6.5em,
           minimum height=2em,
           text centered},
    % Define arrow style
    arr/.style={
           ->,
           thick,
           shorten <=2pt,
           shorten >=2pt,}
}
\usepackage[noend]{algorithmic}
\usepackage[noend]{algorithm}
\newcommand{\bfor}{{\bf for\ }}
\newcommand{\bthen}{{\bf then\ }}
\newcommand{\bwhile}{{\bf while\ }}
\newcommand{\btrue}{{\bf true\ }}
\newcommand{\bfalse}{{\bf false\ }}
\newcommand{\bto}{{\bf to\ }}
\newcommand{\bdo}{{\bf do\ }}
\newcommand{\bif}{{\bf if\ }}
\newcommand{\belse}{{\bf else\ }}
\newcommand{\band}{{\bf and\ }}
\newcommand{\breturn}{{\bf return\ }}
\newcommand{\mod}{{\rm mod}}
\renewcommand{\algorithmiccomment}[1]{$\rhd$ #1}
\newenvironment{checklist}{\par\noindent\hspace{-.25in}{\bf Checklist:}\renewcommand{\labelitemi}{$\Box$}%
\begin{itemize}}{\end{itemize}}
\pagestyle{threepartheadings}
\usepackage{url}
\usepackage{wrapfig}
% removing the standard hyperref to avoid the horrible boxes
%\usepackage{hyperref}
\usepackage[hidelinks]{hyperref}
% added in the dtklogos for the bibtex formatting
\usepackage{dtklogos}
%=========================
% One-inch margins everywhere
%=========================
\setlength{\topmargin}{0in}
\setlength{\textheight}{8.5in}
\setlength{\oddsidemargin}{0in}
\setlength{\evensidemargin}{0in}
\setlength{\textwidth}{6.5in}
%===============================
%===============================
% Macro for document title:
%===============================
\newcommand{\MYTITLE}[1]%
   {\begin{center}
     \begin{center}
     \bf
     CMPSC 440\\Operating Systems\\
     Spring 2014
     \medskip
     \end{center}
     \bf
     #1
     \end{center}
}
%================================
% Macro for headings:
%================================
\newcommand{\MYHEADERS}[2]%
   {\lhead{#1}
    \rhead{#2}
    %\immediate\write16{}
    %\immediate\write16{DATE OF HANDOUT?}
    %\read16 to \dateofhandout
    \def \dateofhandout {March 31, 2014}
    \lfoot{\sc Handed out on \dateofhandout}
    %\immediate\write16{}
    %\immediate\write16{HANDOUT NUMBER?}
    %\read16 to\handoutnum
    \def \handoutnum {8}
    \rfoot{Handout \handoutnum}
   }

%================================
% Macro for bold italic:
%================================
\newcommand{\bit}[1]{{\textit{\textbf{#1}}}}

%=========================
% Non-zero paragraph skips.
%=========================
\setlength{\parskip}{1ex}

%=========================
% Create various environments:
%=========================
\newcommand{\PURPOSE}{\par\noindent\hspace{-.25in}{\bf Purpose:\ }}
\newcommand{\SUMMARY}{\par\noindent\hspace{-.25in}{\bf Summary:\ }}
\newcommand{\DETAILS}{\par\noindent\hspace{-.25in}{\bf Details:\ }}
\newcommand{\HANDIN}{\par\noindent\hspace{-.25in}{\bf Hand in:\ }}
\newcommand{\SUBHEAD}[1]{\bigskip\par\noindent\hspace{-.1in}{\sc #1}\\}
%\newenvironment{CHECKLIST}{\begin{itemize}}{\end{itemize}}


\usepackage[compact]{titlesec}

\begin{document}
\MYTITLE{Laboratory Assignment Two: Implementing and Using a File System Traversal Tool}
\MYHEADERS{Laboratory Assignment Two}{Due: February 10, 2014}

\section*{Introduction}
  
The file system is one of the most heavily used components of an operating system.  Figure~1-14 in your textbook shows a
visualization of a file system that starts at the root directory and branches out to individual files.  The Java programming
language --- and many other languages as well --- provide libraries that support the traversal of a file system and the analysis
of files. In this laboratory assignment, you will implement your own file system traversal tool and use it to learn more about
characteristics of the file system used by the Linux operating system.

\section*{File System Traversal}

For this laboratory assignment you should implement a file system traversal tool in the Java programming language (students who
want to use a different programming language should first see the instructor before starting their implementation). Your program
should use the JCommander command-line argument parsing tool, available for download from \url{http://jcommander.org/}, to accept
as input all of the appropriate arguments. The tool should start at the directory specified by the user and then recursively
traverse the file system until it either recursively visits all of files and directories or it recursively visits up to a
chosen maximum number of files and directories. Students should consider using the methods provides by the {\tt java.io.File}
class to perform \mbox{the traversal}.

Your file system traversal program should be able to calculate the minimum, maximum, and average size of the files that are
accessible by recursive traversal from the specified root directory. The program should also be able to report the total number of
files that it analyzed.  In addition to determining the depth and breadth of the file system subject to analysis, the tool should
also output how long it took to perform the traversal. Students can earn extra credit if their traversal program can produce a
visualization of the file system using a program like Graphviz.

\section*{Experimental Study}

Upon completing your file system traversal tool, you should use it to analyze at least five distinct regions of your file system.
Your analysis should investigate all of the metrics that your tool is able to calculate (e.g., file size and the characteristics
of the file system tree). Students should ensure that their analysis uses large representative file system regions so that the results
are as generalizable as is possible. That is, for the purposes of this assignment, you need to run your analysis tool of directory
structures that contain hundreds, or even thousands, of files. You should also use your tool to determine the size of an empty
file on the file system and the size of the Java program that you wrote to perform the traversal.

\section*{Summary of the Required Deliverables}

  This assignment invites you to submit printed and signed versions of the following deliverables: 

  \begin{enumerate}
    \item 
  \end{enumerate}

  It is recognized that not all of the students in the class may be familiar with the Git version control system and thus have some
  difficulty using and configuring the git plugin for Zsh.  Students who do not have access to a Git repository should see the
  instructor so that one can be made available to them for the purposes of completing this assignment. However, please note that you
  are not required to extensively use a Git repository to complete this assignment. Finally, students are strongly encouraged to
  write their laboratory report in \LaTeX.

  In adherence to the honor code, students should complete this assignment on an individual basis. While it is appropriate for
  students in this class to have high-level conversations about the assignment, it is necessary to distinguish carefully between the
  student who discusses the principles underlying a problem with others and the student who produces assignments that are identical
  to, or merely variations on, someone else's work.  As such, deliverables that are nearly identical to the work of others will be
  taken as evidence of violating the \mbox{Honor Code}.  



  \end{document}
