%!TEX root=cs440S2014-lab2.tex
% mainfile: cs440S2014-lab2.tex 
\input{labspre.tex}

\usepackage[compact]{titlesec}

\begin{document}
\MYTITLE{Laboratory Assignment Two: Implementing and Using a File System Traversal Tool}
\MYHEADERS{Laboratory Assignment Two}{Due: February 10, 2014}

\section*{Introduction}
  
The file system is one of the most heavily used components of an operating system.  Figure~1-14 in your textbook shows a
visualization of a file system that starts at the root directory and branches out to individual files.  The Java programming
language --- and many other languages as well --- provide libraries that support the traversal of a file system and the analysis
of files. In this laboratory assignment, you will implement your own file system traversal tool and use it to learn more about
characteristics of the file system used by the Linux operating system.

\section*{File System Traverasl}

\section*{Summary of the Required Deliverables}

  This assignment invites you to submit printed and signed versions of the following deliverables: 

  \begin{enumerate}
    \item 
  \end{enumerate}

  It is recognized that not all of the students in the class may be familiar with the Git version control system and thus have some
  difficulty using and configuring the git plugin for Zsh.  Students who do not have access to a Git repository should see the
  instructor so that one can be made available to them for the purposes of completing this assignment. However, please note that you
  are not required to extensively use a Git repository to complete this assignment. Finally, students are strongly encouraged to
  write their laboratory report in \LaTeX.

  In adherence to the honor code, students should complete this assignment on an individual basis. While it is appropriate for
  students in this class to have high-level conversations about the assignment, it is necessary to distinguish carefully between the
  student who discusses the principles underlying a problem with others and the student who produces assignments that are identical
  to, or merely variations on, someone else's work.  As such, deliverables that are nearly identical to the work of others will be
  taken as evidence of violating the \mbox{Honor Code}.  



  \end{document}
